\documentclass{article}
\usepackage{graphicx}
\usepackage[bulgarian]{babel}% Required for inserting images

\title{\huge Самостоятелна работа \\ GENB045 Математически анализ}
\author{\LARGE Владимир Стоянов F108982 \\ \LARGE "Нов Български Университет"}
\date{\LARGE 04.06.2024г.}

\begin{document}

\maketitle

\Large
\begin{center}
    \section*{Задача 01}
\end{center}
\newpage
Главата е посветена на основните теореми за диференциално смятане. Дадени са дефиниции за локален минимум, локален максимум и критични точки. Подробно и нагледно е показан алгоритъм за изследване на функция в интервал, в който тя е непрекъсната (диференцуема). Доказана е теоремата на Ферма и са приведени нейните следствия. Определени са критериите за намиране на локален екстремум, интервалите на нарастване и намаляване на функции. Доказани са теоремите на Коши и Лагранж. Всички теореми и техните доказателства са илюстрирани с множество нагледни примери. Главата завършва с примери и задачи, които целят да затвърдят получените знания и да създадат трайни навици при решаване на подобни задачи, които се срещат много често в практиката.

\end{document}
