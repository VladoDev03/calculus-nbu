\documentclass{article}
\usepackage{graphicx}
\usepackage[bulgarian]{babel}% Required for inserting images

\title{\huge Самостоятелна работа \\ GENB045 Математически анализ}
\author{\LARGE Владимир Стоянов F108982 \\ \LARGE "Нов Български Университет"}
\date{\LARGE 04.06.2024г.}

\begin{document}

\maketitle

\Large
\begin{center}
    \section*{Задача 03}
\end{center}
\newpage
\begin{center}
    Ще определим интервалите, в които функцията \[ y = x^\frac{2}{3}(x-2)^2 \] монотонно расте и монотонно намалява, както и точките, в които тя има локален минимум и локален максимум.
\end{center}
\vspace{0.5cm}
(1.) $x^\frac{2}{3} $ и $ (x-2)^2 \in (-\infty, \infty) \Rightarrow y $ е дефиниран \linebreak за всяко $ x $
\vspace{0.5cm} \\
Опростяваме
\[ y = x^\frac{2}{3}(x-2)^2 = \]

\[ = x^\frac{2}{3}(x^2-4x+4) = x^{2\frac{2}{3}} - 4^{1\frac{2}{3}} + 4x^\frac{2}{3} \]

\vspace{0.5cm}
(2.) Намираме първата производна
\[ y' = 2^\frac{2}{3}x^{1\frac{2}{3}} - \frac{20}{3}x^\frac{2}{3} + \frac{8}{3}x^{-\frac{1}{3}} \]

\vspace{0.5cm}
(3.) Намираме точките, в които функцията има \linebreak екстремуми
\[ \frac{8}{3}\sqrt[3]{x^5} - \frac{20}{3}\sqrt[3]{x^2} + \frac{8}{3\sqrt[3]{x}} = 0 \]

\vspace{0.5cm}
Умножаваме по $ \frac{3}{4} $ и се получава:

\[ 2\sqrt[3]{x^5} - 5\sqrt[3]{x^2} + \frac{2}{\sqrt[3]{x}} = 0 \]

\newpage
Полагаме $ \sqrt[3]{x} = u $ и се получава:
\[ 2u^5 - 5u^2 + \frac{2}{u} = 0 \]

\vspace{0.5cm}
Умножаваме по $ u $ и се получава:
\[ 2u^6 - 5u^3 + 2 = 0 \]

\vspace{0.5cm}
Полагаме $ u^3 = t $ и се получава:
\[ 2t^2-5t+2 = 0 \]

\vspace{0.5cm}
Решаваме квадратното уравнение:
\[ t_{1,2} = \frac{5 \pm \sqrt{25-16}}{4} = \frac{5 \pm 3}{4} \]

\vspace{0.5cm}
Намираме, че:
\[ t_1 = \frac{5 - 3}{4} = \frac{1}{2} \qquad t_2 = \frac{5 + 3}{4} = 2 \]

\vspace{0.5cm}
\[ u_1 = \frac{1}{\sqrt[3]{2}} \qquad u_2 = \sqrt[3]{2} \]

\vspace{0.5cm}
Връщаме се към изходната променлива $x$ и от $ u = \sqrt[3]{x} $ следва $ u^3 = x $
\[ x_1 = \frac{1}{2} \qquad x_2 = 2 \]

\newpage

\vspace{0.5cm}
(4.) Намираме втората производна
\[ y' = \frac{4}{3} \left({2x^\frac{5}{3}} - {5x^\frac{2}{3}} + {2x^{-\frac{1}{3}}}\right) \]

\[ y'' = \frac{4}{3} \left({\frac{10}{3}x^\frac{2}{3}} - {\frac{10}{3}x^{-\frac{1}{3}}} - {\frac{2}{3}x^{-\frac{4}{3}}}\right) \]

\vspace{0.5cm}
(5.) Използвайки точките, в които открихме \linebreak екстремуми, заместваме $x$ във втората производна, за да намерим локалните минимум и максимум
\vspace{0.5cm}

\[
    y''\left(\frac{1}{2}\right) = \frac{4}{3} \cdot \left[ \frac{10}{3} \cdot \left(\frac{1}{2}\right)^{\frac{2}{3}} - \frac{10}{2} \cdot \left(\frac{1}{2}\right)^{-\frac{1}{3}} - \frac{2}{3} \cdot \left(\frac{1}{2}\right)^{-\frac{4}{3}} \right] < 0
\]

\vspace{0.25cm}

\begin{center}
    и
\end{center}

\vspace{0.25cm}
\[
    y''(2) = \frac{4}{3} \cdot \left( \frac{10}{3} \cdot 2^{\frac{2}{3}} - \frac{10}{3} \cdot 2^{-\frac{1}{3}} - \frac{2}{3} \cdot 2^{-\frac{4}{3}} \right) =
\]


\[
    = \frac{4}{3} \cdot \left( \frac{10}{3} \cdot \sqrt[3]{4} - \frac{10}{3 \sqrt[3]{2}} - \frac{2}{3 \sqrt[3]{2^4}} \right) > 0
\]

\vspace{1cm}
Следователно при $ x = \frac{1}{2} $ функцията има локален максимум, а при $ x = 2 $ - локален минимум.
\newpage
\begin{center}
    Отговор
\end{center}

\vspace{1cm}

\begin{center}
    При $x \in (-\infty, \frac{1}{2})$ монотонно расте \\
    \vspace{1cm}
    При $x \in (\frac{1}{2}, 2)$ монотонно намалява \\
    \vspace{1cm}
    При $x \in (2, \infty)$ монотонно расте \\
    \vspace{1cm}
    При $x = \frac{1}{2}$ локален максимум \\
    \vspace{1cm}
    При $x = 2$ локален минимум \\
    \vspace{1cm}
    \vspace{2cm}
    \includegraphics[scale=1]{graphic.png}
\end{center}

\end{document}
